\section{Project Feasibility }
\subsection{Operational Feasibility}
This section measures how well the proposed solution meets the user requirements
of the system to solve the issues in the existing system. Therefore, it is necessary to
meet desired requirements to be operationally feasible. 
At present, most of the activities that are related to camping/hiking planning are done offline, or through social media posts which is time-consuming and a waste of money. 
By conducting a public survey, we found out that,
\begin{itemize}
\item People are willing to use such a platform.
\item Guides and campers are interested in using this platform.
\item Rental shops are open to come on board.
\end{itemize}
From our platform, we have proposed to classify the relevant shops with respect to their destination. Also, rental services and guides both will be connected through one platform so it will be really easy for the users.

Our solution is a web application accessed through the Internet. So, to use and operate the system users must have,
\begin{itemize}
\item Internet connection.
\item A mobile phone or a computer with a recent version of a web browser installed.
\item Basic knowledge of IT and using the internet.
\item Intermediate level of knowledge in online and card-based payments for payments.
\end{itemize}

Because there is no need to provide special training to use the system or any
dedicated HR resources to maintain the system, the project is operationally feasible



\subsection{Technical Feasibility}
This section measures the flexibility of the practical implementation of building of our
web-based platform by using the selected technical solutions. The main deliverable of
this project is a web app built using HTML, CSS and JS in the frontend and PHP in the
backend which utilizes MySQL as the datastore. 
The platform relies on; Google Maps API to show locations, a payment gateway to accept payments and also uses SMTP to send
emails.
While developing, we’ll be using Figma for UI/UX prototyping, and Draw.io and Lucidchart for diagramming and modeling. In addition to that, GitHub is used as the code collaborative and version control tool.
The team have to gain adequate technical knowledge before and while building the system. It is a tremendous task, but the timeline allows us to gain a considerable amount of technical knowledge before starting the actual development work and the rest while
building the system itself.

\subsection{Economical Feasibility}
The system could be easily hosted in any cloud provider as we would not be using any
proprietary or vendor-locked technologies.
The technologies we’re using are either open source or freely available, so we won't be needing any cost for that. But deployment, maintenance require some amount of cost, but it will not be a huge amount, hence we can say that this project is economically feasible.

\subsection{Schedule Feasibility}
We started our project in July and now we've come to the system deployment as previously planned.

\subsection{Legal and Ethical Feasibility}
This section measures the legal and ethical issues we would be facing when building and implementing our platform and our countermeasures for that.
\begin{itemize}
    \item All user identity is verified when registering and each login.
    \item Personal information will be protected and user passwords will be hashed.
    \item The service records collected and stored related to users will not be exposed to a third party under any circumstances.
    \item Application is built following the license of the software technologies that are used.
    \item Building and deployment of the system is done according to the laws of Data protection and privacy act and Electronic transactions act.
\end{itemize}